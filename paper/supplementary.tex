\documentclass[11pt]{article}

\input{../latex/preamble}

% load acronyms from external file
% --- Acronyms ---
\DeclareAcronym{soh}{
  short = SoH ,
  long  = State of Health
}
\DeclareAcronym{soc}{
  short = SoC ,
  long  = State of Charge
}
\DeclareAcronym{rul}{
  short = RUL ,
  long  = Remaining Useful Life
}
\DeclareAcronym{eol}{
  short = EoL ,
  long  = End of Life
}
\DeclareAcronym{dod}{
  short = DoD ,
  long  = Deapth of Discharge
}
\DeclareAcronym{lfp}{
  short = LFP ,
  long  = Lithium Iron Phosphate
}
\DeclareAcronym{pca}{
  short = PCA ,
  long  = Principal Component Analysis
}
\DeclareAcronym{pls}{
  short = PLS ,
  long  = Partial Least Squares
}
\DeclareAcronym{nn}{
  short = NN ,
  long  = Neural Network
}
\DeclareAcronym{iqr}{
  short = IQR ,
  long  = Interquartile Range
}
\DeclareAcronym{std}{
  short = STD ,
  long  = Standard Deviation
}
\DeclareAcronym{pcc}{
  short = PCC ,
  long  = Pearson Correlation Coefficient
}

% --- Metrics ---
\DeclareAcronym{mse}{
  short = MSE ,
  long  = Mean Squared Error ,
  extra = {The average of squared differences between predicted and actual values, used to measure model accuracy.}
}
\DeclareAcronym{mae}{
  short = MAE ,
  long  = Mean Absolute Error ,
  extra = {The average of absolute differences between predicted and actual values, used to assess prediction accuracy.}
}
\DeclareAcronym{rmse}{
  short = RMSE ,
  long  = Root Mean Squared Error ,
  extra = {The square root of the mean squared error, providing a measure of prediction error in the same units as the target.}
}
\DeclareAcronym{r2}{
  short = R\textsuperscript{2} ,
  long  = Coefficient of Determination ,
  extra = {A metric indicating how well a model explains the variability of the target variable, ranging from 0 to 1.}
}


\begin{document}

\title{Supplementary Material: Lightweight Single-Cycle Prognostics for Li-ion Batteries}
\author{Gabriel Buzzi Sanches, Carlos Antônio Rufino Junior, Lucas Saraiva Texeira, Hudson Zanin \\
\small Universidade Estadual de Campinas \\
\small \texttt{gabrielbuzzi2@gmail.com}}
\date{\today}

\maketitle

\appendix
\section{Batch Descriptions}

The dataset is organized into three batches, each containing approximately 48 batteries and identified sequentially as Batch 1, Batch 2, and Batch 3 based on the chronological order of their test start dates. Each batch exhibits specific experimental conditions and irregularities, detailed below.

In Batch 1, all batteries were cycled with one-step or two-step charging policies, with charging times ranging from approximately 8 to 13.3 minutes to reach 80\% SOC. Typically, two batteries were tested per policy, except for the 3.6C(80\%) policy. Rest periods of 1 minute were implemented after reaching 80\% SOC during charging, and 1-second rests followed discharging. Cycling continued until batteries reached 80\% of nominal capacity (0.88 Ah). An initial C/10 cycle was performed at the start of each test, and cutoff currents for constant-voltage steps were set to C/50 for both charging and discharging. The internal resistance test used a pulse width of 30 ms. Two automatic computer restarts during testing caused time gaps in the data, and inconsistent temperature control led to variability in the baseline chamber temperature. Tests in channels 4 and 8 failed to start, resulting in missing data, and thermocouples for channels 15 and 16 were accidentally switched. Cycle 1 data was excluded from the MATLAB struct due to an initially high sampling rate, omitted to manage file sizes. Batteries in channels 1, 2, 3, 5, and 6, using 3.6C(80\%) and 4C(80\%) policies, were paused at the end of this batch and resumed in Batch 2, leading to a capacity increase upon resumption. Tests in channels 13, 19, 21, 22, and 31 were terminated before reaching 80\% of nominal capacity.

Batch 2 involved cycling all batteries with one-step or two-step charging policies, with a fixed charging time of 10 minutes to reach 80\% SOC. Typically, one battery was tested per policy, except for the 4.8C(80\%) policy, which included three replicates. Five batteries from Batch 1 (3.6C and 4.0C policies) were resumed in this batch. Cycling continued until batteries reached 75\% of nominal capacity (0.88 Ah). Rest periods of 5 minutes were implemented after reaching 80\% SOC during charging and after discharging. An initial C/10 cycle was performed at the start of each test, with cutoff currents for constant-voltage steps set to C/50 for both charging and discharging. The internal resistance test used a pulse width of 30 ms. A computer restart around cycle 250 for most policies caused an approximately 8-hour rest period, resulting in capacity spikes. Another restart near the end of testing affected one battery (Channel 3, EL150800460623), potentially showing a capacity spike near end-of-life. Thermocouples in channels 7 and 21 detached during cycling, and those in channels 15 and 16 were switched. Data for 3.6C(80\%) and 4C(80\%) policies in channels 1, 2, 3, 5, and 6 are continuations from Batch 1, not new experiments. The battery in Channel 10 (EL150800460605) exhibited rapid degradation, possibly due to a defect, though no clear evidence was found.

In Batch 3, all batteries were cycled with two-step charging policies, with a fixed charging time of 10 minutes to reach 80\% SOC. Multiple batteries (3–8) were tested per policy. Cycling continued until batteries reached 80\% of nominal capacity (0.88 Ah). Four 5-second rest periods were implemented: after reaching 80\% SOC during charging, after the internal resistance test, before discharging, and after discharging. A final C/10 cycle was performed at 80\% of nominal capacity, with cutoff currents for constant-voltage steps set to C/20 for both charging and discharging. The internal resistance test used a pulse width of 33 ms. Some batteries experienced open-circuit voltage errors due to the internal resistance test, causing temporary pauses in cycling. Tests in channels 33 and 41 were terminated before reaching 80\% of nominal capacity. The battery in channel 46 exhibited noisy voltage profiles, likely due to an electronic connection error.

\subsection{Dataset Limitations}

The dataset, while robust for controlled experimental analysis, presents several limitations that impact the development of machine learning models for estimating the remaining useful life (RUL) and state of health (SOH) of lithium-ion batteries. All analyzed cells were cycled in a laboratory under highly controlled conditions, with continuous full charge-discharge cycles (from 0\% to 100\% state of charge and back to 0\%) and minimal idle periods between cycles. This setup facilitates systematic analysis and reproducibility but does not accurately reflect real-world operational profiles, such as those in electric vehicles or stationary energy storage systems, which often involve extended rest periods and partial, intermittent charge-discharge cycles. These differences affect both the collected signals (e.g., voltage and current profiles over time) and the cell aging process itself. As noted in the literature, cells subjected to such laboratory cycling protocols tend to exhibit shorter lifespans compared to those in real-world conditions~\cite{geslin2025}, limiting the direct applicability of developed models to operational systems.

Additionally, the dataset is constrained by its focus on a single battery chemistry (LFP/graphite) from a single manufacturer, which restricts the generalizability of machine learning models to other battery types, such as nickel manganese cobalt (NMC) or nickel cobalt aluminum oxide (NCA) chemistries, that are prevalent in various applications. The experiments were conducted at a fixed temperature of 30°C, which does not capture the impact of temperature variations commonly encountered in real-world scenarios, such as extreme heat or cold, which significantly influence battery degradation rates. The emphasis on fast-charging protocols, while valuable for studying high-rate performance, may not adequately represent slower charging scenarios or dynamic load profiles typical in practical applications, potentially limiting the models' ability to predict RUL and SOH under diverse operating conditions. Furthermore, the inconsistencies in temperature measurements due to unreliable thermocouple contacts and the presence of experimental irregularities (e.g., computer restarts, missing data, and noisy voltage profiles) introduce noise and gaps in the data, which may challenge the robustness of machine learning models, particularly those relying on precise temporal or thermal features for accurate predictions.

This dataset provides a robust foundation for applying machine learning techniques to estimate \ac{soh} and \ac{rul}, capturing a range of fast-charging conditions and their impacts on battery degradation, but its limitations necessitate careful consideration when extending models to real-world applications.

\section{Limitations and Future Research Directions}

Several limitations must be acknowledged in the current optimization approach. The use of a 10\% subset (\cref{fig:subset_distributions}) for hyperparameter optimization, while computationally efficient, may not fully capture the complete data distribution complexity. Additionally, the focus on full-life training data, while comprehensive for model evaluation, may not reflect real-world scenarios where models must perform early-life predictions with limited degradation history.

A fundamental limitation lies in the requirement for complete CCCV cycle data for feature extraction. This dependency significantly restricts the practical applicability of the developed models, as such data is typically unavailable during normal battery operation and can only be obtained during scheduled diagnostic procedures. Future research should prioritize the development of feature extraction methods that can operate on partial cycle data, operational load profiles, or other readily available signals to enable continuous prognostic monitoring without service interruption.

The identified overfitting tendencies, particularly for \ac{rul} prediction, highlight the need for regularization strategies beyond hyperparameter optimization. Future work should investigate ensemble approaches that combine multiple models or incorporate physics-informed constraints to improve generalization while maintaining predictive accuracy. The development of hybrid models that combine the generalization properties of simpler methods with the predictive power of complex ensemble approaches could address the observed overfitting issues.

The consistent challenges observed in \ac{rul} prediction across all models indicate a need for alternative problem formulations, such as classification-based approaches that predict remaining life ranges rather than precise cycle counts, or multi-task learning frameworks that jointly optimize \ac{soh} and \ac{rul} estimation to leverage their inherent relationships. Additionally, investigating domain-specific regularization techniques that incorporate known battery aging physics could improve model robustness and reduce the sensitivity to training data distribution variations.

\printacronyms[name=List of Acronyms]

\bibliography{../latex/references}

\end{document}
