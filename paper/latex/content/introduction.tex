\section{Introduction}

The transition to renewable energy and the rapid adoption of electric mobility have placed \acf{libs} at the center of modern energy storage technologies. Ensuring their safety, reliability, and economic viability requires accurate monitoring of degradation and forecasting of future performance, a field known as battery prognosis. Two key indicators are \acf{soh}, the capacity relative to a new cell, and \acf{rul}, the cycles until a predefined end-of-life. Reliable estimation of these metrics is critical for optimizing battery use, enabling predictive maintenance, and extending system lifetime in applications such as electric vehicles and stationary storage systems.  

Despite significant progress, existing methods for battery prognosis face important challenges. Many \acf{ml} approaches rely on handcrafted features extracted from full charge--discharge cycles, which restricts their applicability to controlled datasets where continuous cycling data are available. More advanced deep learning methods have been proposed to process raw electrochemical signals directly, thereby bypassing manual feature engineering \cite{Che2022, Lv2024}. However, these approaches introduce substantial computational demands and often lack interpretability, which limits their deployment in embedded battery management systems. Furthermore, most studies validate their methods on laboratory datasets collected under highly controlled conditions, often with aggressive fast-charging protocols and uniform cycling patterns \cite{Jafari2024, Liu2025}. Such protocols do not reflect the intermittent and partial cycling profiles seen in real-world applications, reducing the generalizability of the resulting models.  

Another key limitation is the reliance on multiple consecutive cycles to extract reliable prognostic features. Operational systems rarely provide uninterrupted sequences of complete charge--discharge cycles, making such requirements impractical. This highlights the need for models capable of producing robust SoH and RUL estimates from limited information, ideally from a single diagnostic cycle, without assuming continuous access to full degradation histories.

In this work, we address these challenges by developing machine learning models trained on a large, publicly available dataset of 124 commercial LiFePO\textsubscript{4}/graphite cells cycled under fast-charging conditions \cite{sev2019}. Unlike prior studies that depend on sequences of multiple cycles, our methodology demonstrates that accurate inference of both SoH and RUL can be achieved using only the information contained in a single complete cycle. This approach reflects realistic diagnostic scenarios, where a dedicated cycle can be periodically performed to assess battery condition without interrupting normal operation. By extracting statistical features from voltage, current, and temperature signals at the cycle level, we provide a compact and computationally efficient representation suitable for supervised learning. Our contributions are twofold: (i) we show that reliable prognosis does not require long degradation histories for inference, thereby increasing the applicability of ML methods in real-world battery management systems, and (ii) we present a systematic evaluation of feature-based models for SoH and RUL estimation, highlighting the balance between interpretability, efficiency, and predictive performance.  