\documentclass[11pt]{article}

\input{preamble}

\begin{document}

\newacronym{soh}{SOH}{State-of-Health}
\newacronym{rul}{RUL}{Remaining Useful Life}
\newacronym{eol}{EoL}{End-of-Life}
\newacronym{pca}{PCA}{Principal Component Analysis}
\newacronym{pls}{PLS}{Partial Least Squares}
\newacronym{nn}{NN}{Neural Network}
\newacronym{iqr}{IQR}{Interquartile Range}
\newacronym{std}{STD}{Standard Deviation}
\newacronym{pcc}{PCC}{Pearson Correlation Coeficient}
\newglossaryentry{mse}{name=MSE, description={Mean Squared Error, the average of squared differences between predicted and actual values, used to measure model accuracy}}
\newglossaryentry{mae}{name=MAE, description={Mean Absolute Error, the average of absolute differences between predicted and actual values, used to assess prediction accuracy}}
\newglossaryentry{rmse}{name=RMSE, description={Root Mean Squared Error, the square root of the mean squared error, providing a measure of prediction error in the same units as the target}}
\newglossaryentry{r2}{name=R2, description={Coefficient of Determination, a metric indicating how well a model explains the variability of the target variable, ranging from 0 to 1}}

\title{Machine Learning Techniques for Li-ion Battery State-of-Health and Remaining Useful Life Estimation}
\author{Gabriel Buzzi Sanches \\
\small Universidade Estadual de Campinas \\
\small \texttt{gabrielbuzzi2@gmail.com}}
\date{\today}

\maketitle

\begin{abstract}
    Your abstract goes here.
\end{abstract}

\section{Introduction}

% Need to insert the citations to the papers on bibtex

Lithium-ion batteries have emerged as the dominant energy storage technology for applications ranging from electric vehicles to grid-scale storage systems, yet their performance degradation over time remains a critical challenge for operational reliability and economic viability. Accurate estimation of battery State of Health (SoH) and Remaining Useful Life (RUL) is essential for predictive maintenance, warranty management, and ensuring safe operation throughout the battery lifecycle. While numerous machine learning approaches have been proposed to address this challenge, significant gaps remain between laboratory demonstrations and practical deployment in real-world battery management systems.

The complexity of battery degradation mechanisms, which involve interconnected electrochemical, thermal, and mechanical processes, makes accurate prognosis particularly challenging. These degradation patterns manifest differently depending on operating conditions, usage profiles, and environmental factors, necessitating robust modeling approaches that can generalize across diverse scenarios. Recent advances in machine learning have shown promise in capturing these complex relationships; however, critical methodological limitations often compromise their real-world applicability.

A fundamental challenge in battery prognosis research is the reliance on controlled laboratory data that may not represent actual usage conditions. Many existing approaches depend on features extracted from complete charge-discharge cycles performed under constant current conditions—a scenario rarely encountered in practical applications where batteries experience partial, irregular charging patterns and dynamic load profiles. More critically, several methods rely on multiple subsequent constant-current–constant-voltage (CCCV) cycles to build predictive features or model trajectories. Such requirements impose strong assumptions about data availability that are unrealistic in practice, since real-world batteries are not cycled in long, uninterrupted CCCV sequences. In contrast, diagnostic opportunities in the field typically arise during maintenance or scheduled service, where at most a single complete CCCV cycle can be recorded.

Another pervasive issue in the literature is data leakage, where information from test datasets inadvertently influences model training through feature engineering or preprocessing steps. For instance, some methods apply signal decomposition techniques or normalization procedures to entire datasets before splitting them into training and test sets, allowing future information to contaminate the training process. Similarly, approaches that rely on absolute capacity measurements or State of Charge values as input features assume access to ground-truth information that is difficult or impossible to obtain reliably in operational settings, particularly for aged cells where capacity itself is the parameter being estimated.

The feature extraction strategies employed in many studies further limit practical deployment. Methods that require multi-cycle historical data or depend on specific portions of the charge-discharge curve selected based on capacity thresholds introduce circular dependencies and operational constraints. Additionally, the widespread use of constant current discharge data for model validation, while convenient for laboratory studies, does not reflect the variable and unpredictable discharge patterns encountered in real-world applications such as electric vehicles or consumer electronics.

This work addresses these limitations through a comprehensive framework for battery SoH and RUL estimation designed with practical deployment considerations at its core. Using the publicly available dataset from Severson et al. (2019), which contains cycling data from 124 commercial lithium iron phosphate batteries tested under various fast-charging conditions, we develop and validate machine learning models that generalize to unseen cells without requiring personalized training. Our approach employs statistical features extracted from a single complete CCCV cycle, eliminating dependencies on multi-cycle data or absolute capacity measurements. Crucially, this design mirrors how a diagnostic cycle could be performed in practice during regular maintenance, enabling inference without the unrealistic assumption of continuous CCCV cycling. We further implement rigorous data splitting strategies where entire cells are reserved for testing, ensuring no information leakage between training and evaluation sets.

The primary contributions of this work include: (1) a systematic analysis of feature extraction methods that rely only on directly measurable signals without assuming knowledge of absolute capacity or State of Charge, (2) implementation of multiple machine learning models (Tweedie Regressor, K-Nearest Neighbors, Extra Trees, and LightGBM) with careful hyperparameter optimization using cross-validation techniques that prevent data leakage, (3) comprehensive evaluation demonstrating model generalization to completely unseen batteries, and (4) critical assessment of the dataset's limitations and their implications for real-world deployment. By addressing the methodological weaknesses prevalent in existing approaches, this work provides a more realistic assessment of achievable performance in battery prognosis while establishing best practices for future research in this domain.

\section{Dataset Descriptio}

The dataset utilized in this study, originally presented in~\cite{sev2019}, comprises cycling data from 124 commercial lithium iron phosphate (LFP)/graphite batteries. These batteries, manufactured by A123 Systems (model APR18650M1A), have a nominal capacity of 1.1 Ah and a nominal voltage of 3.3 V. The original study aimed to optimize fast charging for lithium-ion batteries, with all batteries cycled to failure under various fast-charging conditions in a controlled environment. In this work, the dataset was accessed using the provided MATLAB struct (.mat) files, which serve as a convenient container for individual cycle data and can be loaded in MATLAB or Python (via the h5py package). Pandas dataframes can also be generated from these structs using provided code.

The batteries were cycled in horizontal cylindrical fixtures on a 48-channel Arbin LBT potentiostat within a forced convection temperature chamber maintained at 30°C. Charging was performed using either one-step or two-step fast-charging policies, denoted as "C1(Q1)-C2," where C1 and C2 represent the constant current rates for the first and second steps, respectively, and Q1 indicates the state-of-charge (SOC, in \%) at which the current switches. The second current step concluded at 80\% SOC, followed by a 1C constant current-constant voltage (CC-CV) charging protocol. The upper and lower voltage cutoffs were set at 3.6 V and 2.0 V, respectively, adhering to the manufacturer's specifications. These voltage limits were enforced across all charging steps, including fast charging, and after several cycles, some batteries reached the upper voltage limit during fast charging, resulting in extended constant-voltage charging periods. All batteries were discharged at a constant current rate of 4C.

The dataset includes measurements of voltage, current, temperature, charge capacity, discharge capacity, and internal resistance for each cycle. The temperature data was collected by attaching type T thermocouples to the exposed cell cans with thermal epoxy (OMEGATHERM 201) and Kapton tape, after removing a small section of the plastic insulation. However, the reliability of temperature measurements may vary due to inconsistent thermal contact between the thermocouple and the battery can, with some thermocouples losing contact during cycling. Internal resistance was measured during charging at 80\% SOC by averaging 10 pulses of ±3.6C, with a pulse width of 30 ms for Batches 1 and 2, or 33 ms for Batch 3.

\inserttable{batch_summary}

% \insertfigure{fig_model}
% \inserttable{tbl_dataset}

\section{\acl{soh} and \acl{rul}}

The development of machine learning models for battery prognosis requires clearly defined target variables, namely \acf{soh} and \acf{rul} These metrics serve as key performance indicators to quantify the degradation state of a battery. The following section outlines the methodology used to calculate these ground-truth labels from the experimental cycling data.

The \ac{soh} of a battery provides a snapshot of its current condition relative to its initial, "healthy" state. It is typically expressed as a percentage of the nominal capacity. For the purpose of this study, the \ac{soh} for the nth cycle, denoted as $SoH_n$, is calculated as the ratio of the maximum discharge capacity observed during that cycle to the rated cell capacity, as shown in \cref{eq:soh}.

\begin{equation}
    \mathrm{SoH}_n = \frac{Q_n}{Q_{\text{rated}}}
    \label{eq:soh}
\end{equation}

where $Q_n$ is the maximum discharge capacity recorded for the nth cycle, and $Q_rated$ is the rated capacity of a new cell.

A critical limitation in determining the ground-truth \ac{soh} in real-world applications is that batteries rarely undergo a full charge or discharge cycle. This makes it challenging to accurately measure the true capacity at any given time. The controlled laboratory setting used to generate this dataset, where full cycles are performed, provides a consistent and reliable method for calculating this critical ground truth.

The \ac{rul} of a battery represents the number of cycles it can endure before reaching its \ac{eol} condition. The \ac{eol} is a predefined failure threshold, commonly set when the battery's capacity drops to 80\% of its initial rated capacity. The \ac{eol} cycle is therefore defined as the cycle number at which the \ac{soh} first falls to 80\% or below.

\begin{equation}
    \mathrm{EoL} = \min \left\{ n \;\middle|\; \mathrm{SoH}_n \leq 0.80 \right\}
    \label{eq:eol}
\end{equation}

The \ac{rul} for any given cycle n, denoted as $RUL_n$, is then calculated by subtracting the current cycle number from the \ac{eol} cycle number, as expressed in \cref{eq:rul} below.

\begin{equation}
    \mathrm{RUL}_n = \mathrm{EoL} - n
    \label{eq:rul}
\end{equation}

A significant challenge associated with using cycle count to define \ac{rul} is its limited applicability to real-world scenarios, where usage is often irregular and does not consist of discrete, well-defined cycles. Furthermore, determining the ground truth \ac{rul} for a particular cycle requires having the full degradation data up to the battery's \ac{eol}, which is a common limitation when training \ac{rul} prediction models from incomplete datasets.

Below, \cref{fig:targets_distribution} shows the distribution of both \ac{soh} and \ac{rul} on the dataset from \cite{sev2019}. As can be seen, neither target exhibits a symmetric distribution. This skewed nature necessitates the use of machine learning models that are robust to non-normal data distributions and can effectively capture the non-linear degradation behavior of batteries.

\insertfigure{soh_and_rul_targets_distribution}
\section{Data Anlysis}
Building upon the dataset description provided in the previous section, this section presents an exploratory data analysis of the cycling data from 124 commercial lithium iron phosphate (LFP)/graphite batteries, as detailed in~\cite{sev2019}. The analysis focuses on characterizing the voltage, current, temperature, and capacity measurements under the fast-charging protocols described earlier, with an emphasis on their evolution over the cells' lifecycle. These insights are critical for understanding battery degradation patterns and informing the development of machine learning models for estimating remaining useful life (RUL) and state of health (SOH). The dataset, organized into three batches with varying fast-charging protocols, provides a comprehensive basis for analyzing the impact of high C-rates and full depth-of-discharge (DoD) protocols, despite the limitations noted, such as controlled laboratory conditions and inconsistent temperature measurements.

The temporal behavior of voltage, current, and temperature during a representative charge-discharge cycle is illustrated in \cref{fig:voltage_current_temperature}, using the first cycle of the 32nd cell from Batch 2 as an example. This cell follows a 5.2C(50\%)-4.25C protocol, where charging involves a constant current (CC) of 5.2C until 50\% state of charge (SOC), followed by 4.25C until 80\% SOC, as described in the dataset's experimental setup. After a 5-minute rest period, a 1C CC step continues until the upper voltage cutoff of 3.6 V, transitioning to constant voltage (CV) charging with naturally decreasing current until 100\% SOC. Discharge occurs at a constant 4C current until the lower voltage cutoff of 2.2 V, followed by CV discharge until full discharge, adhering to the C/50 cutoff current for Batch 2. The temperature profile, shown in \cref{fig:voltage_current_temperature}(c), reveals peaks during both charging and discharging, with the rate of temperature increase correlating with current intensity. During the rest period, the temperature drops, and during the 1C CC charging phase, it stabilizes, likely due to balanced heat generation and dissipation in the 30°C chamber. This pattern is consistent with the discharge phase, where temperature rises with applied current and stabilizes during CV discharge. The correlation between temperature and the integral of current (i.e., capacity exchange) suggests that temperature is a potential indicator of SOC, as further evidenced in \cref{fig:temperature_and_capacities}, where charge and discharge capacities align closely with temperature trends during their respective phases.

\insertfigure{data_analysis/voltage_current_temperature}
\insertfigure{data_analysis/temperature_and_capacities}

The evolution of SOH across the cells' lifecycle, as defined in the dataset description, is depicted in \cref{fig:all_cell_soh}. Most cells start with an SOH near 100\% and exhibit a gradual decline until a characteristic "knee point", after which the degradation rate accelerates. This behavior is consistent across all batches, despite variations in rest periods and cutoff currents. The cycle life distribution, shown in \cref{fig:cycle_life_distribution}, indicates that most cells reach end-of-life (EoL) around 700 cycles, lower than typical for 1.1 Ah LFP cells, likely due to the aggressive 4C discharge and full DoD protocols described earlier. As noted in the dataset limitations, this accelerated aging may not reflect real-world conditions. An anomaly around cycle 250 in Batch 2, caused by an 8-hour rest period due to a computer restart, manifests as capacity spikes, introducing outliers in voltage, current, and temperature measurements. These outliers are removed during data preprocessing to ensure robust model training.

\insertfigure{data_analysis/all_cell_soh}
\insertfigure{data_analysis/cycle_life_distribution}

The impact of aging on voltage and current profiles is examined in \cref{fig:voltage_current_begin_and_end}, which compares these signals at the beginning, middle, and end the life of the 32nd cell from Batch 2 as an example. At the start (\cref{fig:voltage_current_begin_and_end}(a)), the profiles align with the defined fast-charge protocol. By mid-life (\cref{fig:voltage_current_begin_and_end}(b)), changes are subtle, but at EoL (\cref{fig:voltage_current_begin_and_end}(c)), increased internal resistance, measured as described in the dataset, causes the 3.6 V and 2.2 V thresholds to be reached earlier, leading to extended CV phases with reduced current. This results in shorter cycle durations at EoL, reflecting the reduced capacity observed post-knee point in \cref{fig:all_cell_soh}.

\insertfigure{data_analysis/voltage_current_begin_and_end}

A detailed visualization of the fast-charge protocol is provided in \cref{fig:charge_steps}, with color-coded regions highlighting the CC1, CC2, CC3, CV charge, discharge, and rest periods, as described in the experimental setup. For Batch 2, the protocol includes a 5-minute rest after 80\% SOC and post-discharge, with C/50 cutoff currents for CV phases. This structured protocol ensures consistency across cycles, facilitating the analysis of aging effects.

\insertfigure{data_analysis/charge_steps}

The evolution of voltage, current, and temperature across multiple cycles is shown in \cref{fig:signals_through_cycles}, using a Batch 2 cell as an example. The most significant changes occur near EoL (cycle 501), where increased internal resistance, as measured at 80\% SOC, leads to earlier CV phases and higher peak temperatures during discharge, likely due to enhanced Joule heating. This aligns with the dataset’s observation of thermocouple variability, which may amplify temperature measurement noise at EoL.

\insertfigure{data_analysis/signals_through_cycles}

To assess the impact of fast-charge protocols on cycle life, a correlation analysis was performed, leveraging the dataset’s range of one-step and two-step policies. \cref{fig:charge_current_and_cycle_life} presents a scatter plot of average charge C-rate versus cycle life, where the average C-rate is calculated as a weighted average of the CC steps up to 80\% SOC (e.g., for a 4C(50\%)-5C protocol, $(4 \times 50 + 5 \times (80-50))/80$). A Pearson correlation coefficient of $R = -0.61$ indicates a moderate negative correlation, confirming that higher charge currents, as tested across all batches, accelerate aging, consistent with the dataset’s high C-rate conditions.

\insertfigure{data_analysis/charge_current_and_cycle_life}

Further analysis in \cref{fig:charge_steps_area_diff} evaluates the difference in "area" (C-rate multiplied by SOC duration) between the second and first CC steps (e.g., for a 4C(50\%)-3C protocol, $3 \times (80-50) - 4 \times 50$). A Pearson correlation coefficient of $R = 0.38$ suggests a weak positive correlation, indicating that the order and duration of fast-charge steps, as varied in Batches 1 and 3, have minimal impact on cycle life compared to the overall charge current magnitude.

\insertfigure{data_analysis/charge_steps_area_diff}

In summary, the exploratory data analysis, grounded in the dataset’s detailed measurements and experimental conditions, reveals key trends in LFP battery behavior under fast-charging protocols. Temperature strongly correlates with capacity exchange, supporting its use as an SOC indicator, though measurement reliability is limited by thermocouple inconsistencies. SOH degradation accelerates post-knee point, with cycle life averaging 700 cycles due to aggressive testing conditions, as noted in the dataset limitations. Aging significantly alters voltage and current profiles, with increased internal resistance driving earlier CV phases and shorter cycles at EoL. Correlation analyses confirm that higher charge currents reduce cycle life, while step order has minimal impact. These findings, despite the dataset’s constraints (e.g., single chemistry, controlled environment), provide a robust foundation for developing machine learning models to predict RUL and SOH, leveraging the observed relationships between measurable signals and degradation patterns.

% \inserttable{tbl_dataset}

\section{Feature Extraction}

% Introducing the purpose and approach
Feature extraction is a pivotal step in estimating the State of Health (\gls{soh}) and Remaining Useful Life (\gls{rul}) of lithium-ion battery cells, particularly for cells unseen by the model during training. Since \gls{soh} and \gls{rul} evolve gradually on a cycle-by-cycle basis, analyzing raw signals sample-by-sample—especially in fast-charging scenarios—is impractical due to noise and high dimensionality. Instead, aggregating information from entire charge-discharge cycles provides a more meaningful representation of battery degradation. While one could concatenate all samples from a cycle and input them directly into a machine learning model, this risks introducing collinear features and spurious correlations from irrelevant signal variations. Alternative dimensionality reduction techniques, such as Principal Component Analysis (\gls{pca}), Partial Least Squares (\gls{pls}), or Kernel \gls{pca}, could transform cycle samples into a lower-dimensional latent space in an unsupervised manner, with the transformed data then used for supervised learning. Similarly, neural network auto-encoders, including those leveraging recurrent architectures or convolutional processing of time-series-turned-images, offer sophisticated transformations by encoding signals into compact embeddings. However, directly feeding raw signals into complex \gls{nn} models, while feasible, demands significant computational resources, hindering deployment on portable devices or electric vehicles. Consequently, the most widely adopted approach in the literature, which we employ here, involves calculating a small set of statistical metrics from voltage, current, and temperature signals per cycle. These metrics serve as features for training supervised machine learning models to estimate \gls{soh} and \gls{rul}, balancing simplicity and effectiveness.

\subsection{Signals Preprocessing}

% Detailing preprocessing steps
Before computing statistical metrics, the voltage, current, and temperature signals require preprocessing to ensure consistency and reliability:

\begin{enumerate}
    \item \textbf{Time gap removal:} Intermittent gaps in data collection were addressed by concatenating signals, as these discontinuities do not impact the cycle-wide statistical metrics targeted in this study.
    \item \textbf{Invalid cycle filtering:} Cycles with durations below 100 seconds (approximately 1.7 minutes) or above 6000 seconds (approximately 1 hour 40 minutes) were excluded, deviating from the typical 40-to-60-minute range and likely indicating collection or annotation errors.
    \item \textbf{Sampling rate standardization:} Variations in sampling rates across cycles, possibly due to hardware adjustments balancing data detail and volume, were normalized to 1Hz using linear interpolation. This ensures uniform representation of cycle segments in the statistical metrics.
\end{enumerate}

\subsection{Statistical Metrics}

% Listing and explaining metrics
Post-preprocessing, the following statistical metrics were computed for each cycle’s voltage, current, and temperature signals:

\begin{enumerate}
    \item \textbf{Mean:} The average signal value over the cycle.
    \item \textbf{Median:} The central value when the signal is sorted by magnitude.
    \item \textbf{Standard Deviation (\gls{std}):} A measure of signal variability.
    \item \textbf{Interquartile Range (\gls{iqr}):} The range between the 25th and 75th percentiles, capturing the middle 50\% spread.
    \item \textbf{Kurtosis:} An indicator of the signal distribution’s tailedness.
    \item \textbf{Differential Entropy:} A measure of signal randomness, computed only for voltage due to infinite values arising in current and temperature calculations, possibly from near-constant segments or noise.
\end{enumerate}

This yielded 16 features per cycle: 6 for voltage, 5 for current, and 5 for temperature, forming a compact yet informative representation of the battery’s state. 

\subsection{Features Processing}

% Addressing outliers and noise
Despite preprocessing, some cycles exhibited outlier feature values, evident in \cref{fig:full_cycle_features_without_filter}. Such anomalies, common in real-world settings with unknown cells, can disrupt model training. For training data, we mitigated these issues while preserving them in test data to assess real-world robustness and avoid data leakage, once both processing steps applied relies on the feature values of the cycles around the cycle being processed. Two processing steps were applied:

\begin{enumerate}
    \item \textbf{Spike removal:} For each feature per cell, a 10-cycle sliding window identified values below the 5th or above the 95th percentile, replacing them with the prior cycle’s value to eliminate spikes.
    \item \textbf{Smoothing:} A first-order Savitzky-Golay filter with a 10-cycle window was used to dampen cycle-to-cycle oscillations, which do not reflect the gradual evolution of \gls{soh} and \gls{rul}.
\end{enumerate}

The processed features, shown in \cref{fig:full_cycle_features}, retain degradation trends while reducing noise and anomalies.

\insertfigure{features_extraction/full_cycle_features_without_filter}
\insertfigure{features_extraction/full_cycle_features}

\subsection{Features Analysis}

% Analyzing feature trends
The processed feature trajectories in \cref{fig:full_cycle_features} reveal distinct aging patterns. Voltage mean decreases while median increases, likely due to heightened variability (\gls{std} and \gls{iqr}) from rising internal resistance. Voltage kurtosis and differential entropy decline, possibly as extended constant voltage phases in aged cells yield more uniform signals. Current mean remains near zero, reflecting balanced charge-discharge durations, while median decreases due to prolonged constant-voltage charging with lower currents. Current \gls{std} drops as aged cells handle high currents less variably, and \gls{iqr} rises then falls, possibly tied to impedance shifts. Temperature features generally increase, driven by greater heat from internal resistance, though kurtosis shows a rise-then-fall pattern, meriting further study.

% Correlation analysis
Pearson correlation coefficients between features and targets (\gls{soh} and \gls{rul}), shown in \cref{fig:correlation_full_cycle_filtered_only}, highlight strong relationships, especially among voltage features, making them key candidates for modeling. Spearman coefficients in \cref{fig:spearman_full_cycle_filtered_only} often exceed Pearson values, suggesting non-linear dependencies.

\insertfigure{features_extraction/correlation_full_cycle_filtered_only}
\insertfigure{features_extraction/spearman_full_cycle_filtered_only}

\subsection{Feature Selection}

% Explaining feature selection
Given the 16 extracted features, not all contribute equally to the estimation of \gls{soh} and \gls{rul}. Some may introduce noise or redundancy, particularly in models that lack built-in feature selection mechanisms. While bivariate correlation analysis can identify individual linear relationships with the target, it fails to capture more complex, multivariate interactions.

To address this, we applied a feature selection strategy based on feature importance derived from a Random Forest model. Specifically, we used a Random Forest with 10 estimators — a lightweight yet sufficiently robust configuration that balances stability and computational efficiency. This choice helps mitigate the sensitivity of single decision trees to data splits, providing more reliable importance estimates without incurring high computational cost.

We further enhanced the robustness of the feature importance ranking by using Group K-Fold cross-validation with 10 folds, where each group corresponds to an individual cell. The training set consists of 99 unique cells, allowing for a meaningful and well-balanced 10-fold split. This ensures that the training and validation sets remain group-independent, preventing data leakage between folds. Using 10 folds allows each cell to be evaluated multiple times across different validation splits, increasing the reliability of the aggregated feature importance scores.

In each fold, we trained the Random Forest on the training partition and recorded the feature importance. These importances were then summed across all folds and averaged to produce a final ranking. The top-ranked features were selected for subsequent modeling. This process helps retain the most informative degradation indicators while discarding less relevant or redundant ones, ultimately improving the predictive performance of downstream models by accounting for non-linear and multi-feature interactions.

\section{Models}

Different types of algorithms have distinct properties and are capable of capturing statistical relationships between input (predictor) and output (target) variables in various ways. It is not possible to determine in advance which algorithm will be more or less suitable for a given problem, as this depends on the relationship between the data and the algorithm's capabilities. Generally, linear models can recognize simpler patterns (linear in nature), while nonlinear models can capture more complex relationships, albeit at the cost of requiring larger amounts of training data.
In addition to the type of model used, the ``settings'' of each model, known as hyperparameters, can also be varied. These are parameters not automatically learned during training but must be manually adjusted beforehand. Tuning these hyperparameters alters the model's properties, affecting the training process and the final inference performance.
In this context, the most common approach in the literature for selecting an algorithm and its hyperparameter settings for a specific problem is experimentation and comparison. Accordingly, this work evaluates a series of algorithms through a systematic hyperparameter tuning process to determine which algorithm best fits the problem's data, achieving the highest inference accuracy. The algorithms evaluated are listed below, followed by a brief explanation of their functionality, with the hyperparameter optimization process described in detail in subsequent sections.

\subsection{Elastic Net}

\subsection{KNN}

\subsection{Random Forest}

Random Forest is an ensemble method that combines multiple decision trees using bootstrap aggregating (bagging). Each tree is trained on a different bootstrap sample of the data, and at each split, only a random subset of variables is considered.
The final prediction is the average of all tree predictions: \begin{equation} \hat{y} = \frac{1}{B} \sum_{b=1}^{B} T_b(\mathbf{x}) \end{equation}
where $B$ is the number of trees, and $T_b$ is the $b$-th tree. This approach significantly reduces variance compared to individual trees while maintaining low bias. Random Forest offers strong out-of-the-box performance, is robust to outliers, and provides measures of variable importance.

\subsection{LightGBM}

LightGBM (Light Gradient Boosting Machine) is a gradient boosting framework developed by Microsoft that uses tree-based algorithms. Unlike other boosting methods that grow trees level-wise, LightGBM grows trees leaf-wise, expanding the leaf that most reduces the loss.
The model combines multiple weak trees sequentially: 
\begin{equation} 
    F_m(\mathbf{x}) = F_{m-1}(\mathbf{x}) + \gamma_m h_m(\mathbf{x}) 
\end{equation}
where $h_m$ is the $m$-th tree, and $\gamma_m$ is its associated weight. LightGBM offers high computational efficiency, low memory usage, and good accuracy, making it particularly effective for large datasets. It includes optimizations such as continuous variable discretization and efficient parallelization.


% \insertfigure{fig_model}
% \inserttable{tbl_dataset}

\section{Model Training and Optimization}
\label{sec:optimiation}
% Introducing the purpose and approach
This section describes the methodology for training and optimizing machine learning models to estimate the \acf{soh} and \acf{rul} of lithium-ion batteries, utilizing the features extracted from the dataset detailed in prior sections. Machine learning models are mathematical frameworks that identify patterns in data to predict outcomes, here mapping cycle features to \ac{soh} or \ac{rul}. Separate models were trained for \ac{soh} and \ac{rul} to avoid the complexity of joint estimation, focusing on generalization to cells not included in the training data, a critical requirement for applications such as battery management systems in electric vehicles or energy storage.

\subsection{Data Preparation}

% Describing train-test split
To ensure models generalize to unseen cells, the dataset of 124 cells was partitioned into training and test sets by randomly assigning entire cells, preventing any cell’s data from appearing in both sets. This approach mitigates data leakage, where models could exploit specific cell behaviors, leading to overly optimistic performance estimates. As presented in \cref{tab:splits_cells_summary}, the training set comprises 99 cells (80,618 cycles), and the test set includes 25 cells (18,480 cycles), adhering to an approximate 80:20 split, a standard practice to balance training data availability with robust evaluation.

\begin{table}[h]
    \centering
    \begin{tabular}{lrr}
        \toprule
        \textbf{Partition} & \textbf{No. of Cells} & \textbf{Total Samples (Cycles)} \\
        \midrule
        Training & 99 & 80618 \\
        Test     & 25 & 18480 \\
        \midrule
        \textbf{Total} & \textbf{124} & \textbf{99098} \\
        \bottomrule
    \end{tabular}
    \caption{Division of cells and total samples (cycles) into training and test sets.}
    \label{tab:splits_cells_summary}
\end{table}

% Describing feature scaling
Features were standardized using Standard Scaling, transforming each feature to have a mean of zero and a \acf{std} of one, based on statistics derived from the training set. Standardization ensures that features with different scales (e.g., voltage mean in volts vs. unitless kurtosis) contribute equally to model training, preventing bias toward larger-magnitude features. The scaling parameters from the training set were applied to the test set to maintain consistency, mirroring real-world scenarios where new data are processed using established parameters.

\subsection{Model Training and Evaluation}

% Describing model training
Model training entails optimizing internal parameters (weights) to minimize prediction errors on the training data. In this study, models learn to map cycle features (e.g., voltage mean, temperature \ac{std}) to \ac{soh} or \ac{rul}, minimizing the \acf{mse}, which quantifies the average squared difference between predicted and actual values, penalizing larger errors more heavily. Ground truth \ac{soh} and \ac{rul} values from all cycles of the 99 training cells were utilized, leveraging the dataset’s full-life data. In practical applications, full-life data may be unavailable, necessitating alternative approaches: (1) personalized models trained on early cycles of a specific cell for its future predictions, or (2) general models trained on early cycles from multiple cells, applicable to any cell at any stage. These methods are challenging, as early-cycle data may not capture aging patterns, a significant hurdle in battery research. Given the availability of full-life data, this study uses all cycles from training cells to assess feature and model performance for unseen cells, aligning with the objective of generalization.

% Describing model evaluation
Model performance was evaluated by generating predictions for the test set (25 unseen cells) and comparing them to ground truth \ac{soh} and \ac{rul} values using three metrics:

\begin{itemize}
    \item \textbf{\acf{mae}:} The average absolute difference between predicted and actual values, expressed in the target’s units (e.g., percentage for \ac{soh}, cycles for \ac{rul}).
    \begin{equation}
        \mathrm{MAE} = \frac{1}{n} \sum_{i=1}^{n} \left| y_i - \hat{y}_i \right|
        \label{eq:mae}    
    \end{equation}
    
    \item \textbf{\acf{rmse}:} The square root of \ac{mse}, emphasizing larger errors while maintaining the target’s units.
    \begin{equation}
        \mathrm{RMSE} = \sqrt{\frac{1}{n} \sum_{i=1}^{n} (y_i - \hat{y}_i)^2}
        \label{eq:rmse}
    \end{equation}
    
    \item \textbf{\acf{r2}:} A metric from 0 to 1, indicating the proportion of variance in the target explained by the model, with 1 representing perfect predictions.
    \begin{equation}
        R^2 = 1 - \frac{\sum_{i=1}^{n} (y_i - \hat{y}_i)^2}{\sum_{i=1}^{n} (y_i - \bar{y})^2}
        \label{eq:r2}
    \end{equation}
\end{itemize}

For inference, models require voltage, current, and temperature signals from a single full charge-discharge cycle, sampled at 1 Hz or higher, processed into the 15 features described previously.

\subsection{Hyperparameter Optimization}

% Defining hyperparameters and optimization approach
Machine learning models rely on hyperparameters—configurable settings that define their structure or learning process, such as the number of trees in a Random Forest or the learning rate in a neural network. Unlike model weights, hyperparameters are set prior to training and typically determined empirically. To automate this process, the Tree-structured Parzen Estimator (TPE) \cite{tpe_2011}, a Bayesian optimization algorithm, was employed. TPE iteratively samples from a predefined hyperparameter space, balancing exploration of new configurations with exploitation of previously successful ones.

To balance predictive accuracy and model generalization, a custom cost function was adopted, combining validation error and the generalization gap. The objective to minimize is defined as:

\begin{equation}
    \mathcal{L} = \mathrm{RMSE}_{\text{val}} + \frac{\left| \mathrm{RMSE}_{\text{train}} - \mathrm{RMSE}_{\text{val}} \right|}{\mathrm{RMSE}_{\text{val}}}
\end{equation}

where \(\text{RMSE}_{\text{val}}\) is the root mean square error on the validation folds, as defined by \cref{eq:rmse}, and the second term, \(\frac{|\text{RMSE}_{\text{train}} - \text{RMSE}_{\text{val}}|}{\text{RMSE}_{\text{val}}}\), represents the relative generalization gap between training and validation errors. This formulation penalizes both high validation errors and large discrepancies between training and validation performance, effectively discouraging overfitting while prioritizing predictive accuracy.

% Explaining cross-validation setup
Trials were evaluated using 5-fold Group K-Fold cross-validation. The 99 training cells were partitioned into 5 non-overlapping groups (based on cell ID), ensuring that all cycles from a given cell remained within the same fold to prevent data leakage. The final loss score is computed as the average of the defined loss across the 5 validation folds. \cref{fig:kfold} below illustrates the 5-fold cross-validation process adopted.

% Detailing early stopping strategy
To improve optimization efficiency and prevent unnecessary trials, an early stopping strategy was employed. If the cost does not improve for a fixed number of consecutive trials (patience = 10), the optimization process is halted. This ensures that exploration ceases when no meaningful improvement is observed.

\insertfigure{optimization/kfold}

% Data subset for optimization
\insertfigure{optimization/subset_distributions}

% Summarizing the methodology
This methodology leverages full-life data from training cells to develop models that generalize to unseen cells, addressing the challenges of \ac{soh} and \ac{rul} estimation. Through standardized features, multi-object optimization and hyperparameter optimization via TPE and cross-validation, the approach ensures robust predictions avoiding artificially inflated metrics due to overfitting.

\section{Results}

\subsection{Optimization Results}

\cref{tab:soh_training_metrics,tab:rul_training_metrics} below show the results of the optimization process for both \gls{soh} and \gls{rul}, respectively. It can be seen that the models produced good values of validation \gls{rmse}, which is the average of the \gls{rmse} on each fold of cross validation, while maintaining low relative gaps which is a good indicative of non-overfitting models. For \gls{rul} models, authough it was harder to keep the relative gap values low, which indicates models for that target with the features extracted on this work are more prone to overfitt the training data.

\inserttable{soh_optim}
\inserttable{rul_optim}

% Insert table with the best hyperparams for each model

\subsection{Test Results}

The results of each model type using the optimzal hyperparam set are shown on \cref{tab:soh_metrics,tab:rul_metrics} for \gls{soh} and \gls{rul}, respectively. The performance drop between using 16 and 4 features were relatively low indicating models with only the most important features identifies by the Tree-based ensamble recribed on \cref{sec:feature_extraction} is feasible.

The overall performance of all models is satisfatory with \gls{rmse} values below 0.80\% for \gls{soh} using a LGBMRegressor for 75\% of the test cells and below 59.91 for \gls{rul} using an ExtraTreesRegressor for 75\% of the test cells. \gls{r2} values for 75\% of the test cells were above 0.97 and 0.96 for \gls{soh} and \gls{rul}, respectively. Indicating high accordance between the models estimations and references values.

\inserttable{soh_metrics}
\inserttable{rul_metrics}


\cref{fig:best_16_soh_model,fig:best_16_rul_model,fig:best_4_soh_model,fig:best_4_rul_model} below show in detail the scatter plots of all reference vs. estimated values of all test cells for the best model in each case: \gls{soh} and \gls{rul} and unsing all 16 features and only 4 features. It can be confirmed that the model works realy well for most data while some points are more distant from the indeal fit line. For \gls{soh} models under and over estimate values while for \gls{rul} models tend to under estimate the values. For the \gls{rul} models with 16 and 4 features it can be noticed that a similar set of points diverge from the ideal regression line, we identified that this points corresponde to the same cell of the test set. From the plot it can be seen that this cells is the only with almost 2000 cycles cyclelife, this divergence of the model might be happening because data with \gls{rul} values that high are more scarce on the training set as well due to the fact fill cells cycle that long on the dataset used.


\insertfigure{results/best_16_soh_model}
\insertfigure{results/best_4_soh_model}
\insertfigure{results/best_16_rul_model}
\insertfigure{results/best_4_rul_model}
\section{Conclusion}

This work investigated machine learning techniques for estimating the \acf{soh} and \acf{rul} of \acf{libs} using the large-scale dataset provided by Severson et al.~\cite{sev2019}. The dataset, comprising 124 commercial LiFePO\textsubscript{4}/graphite cells cycled under various fast-charging conditions, enabled a systematic exploration of degradation patterns and their relationship with measurable signals such as voltage, current, and temperature. Through comprehensive preprocessing and the extraction of statistical features, we established a compact yet informative representation of cell behavior that balances interpretability and computational efficiency.  

The exploratory analysis confirmed well-known degradation phenomena, including the characteristic knee point in \ac{soh} evolution, accelerated capacity fade under high charging rates, and the influence of internal resistance growth on voltage and temperature profiles. These findings provide a robust foundation for supervised learning models by linking observable features to battery aging mechanisms.  

Nevertheless, the dataset’s constraints must be acknowledged. The exclusive focus on one chemistry (LFP/graphite), highly controlled laboratory conditions, and continuous full charge--discharge cycles limit the direct applicability of the models to real-world scenarios such as electric vehicles or grid storage, where cycling is partial, irregular, and influenced by environmental variations. The inconsistencies in temperature measurements and experimental irregularities further highlight the need for robustness in model design.

Despite these limitations, the results demonstrate that statistical feature representations of single cycles can capture essential information for battery prognosis. This finding points toward a practical diagnostic framework in which SoH and RUL can be estimated from periodic diagnostic cycles, without requiring continuous monitoring of long cycle histories. Future work will extend this approach to heterogeneous datasets, incorporate physics-informed features to enhance generalization across chemistries and conditions, and evaluate the deployment of the proposed methodology in realistic battery management system environments.  

A limitation of the present approach is the requirement of a dedicated full diagnostic cycle, which may not be available in continuous operation. Future work will evaluate the feasibility of periodic diagnostics (e.g., monthly) and the extension to multiple chemistries such as NMC and NCA.


\bibliography{references}

\end{document}
