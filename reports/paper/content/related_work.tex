\section{Related Works}

Several recent studies have sought to improve the accuracy, generalization, and interpretability of ML-based methods for battery prognosis. \cite{Che2022} proposed a probabilistic LSTM framework that extracts health indicators from partial Q--V curves, demonstrating strong generalization across multiple datasets of LiFePO\textsubscript{4} cells. While their method reduces data requirements through transfer learning, it remains specific to a single chemistry and depends on carefully chosen voltage ranges.  

\cite{Jafari2024} explored measurable features such as discharge time and temperature, integrating \acf{xai} methods to improve interpretability. Although their approach is practical for electric vehicle applications, it was validated on a very limited dataset of only four cells, restricting its generalizability.

To address the lack of physical interpretability in purely data-driven methods,~\cite{Liu2025} proposed a hybrid model combining an incremental internal resistance aging model (IIRAM) with a gated recurrent unit (GRU) network. This approach improved robustness and provided physically meaningful health indicators but remained restricted to constant-current conditions, which rarely occur in real-world systems. Similarly,~\cite{Lv2024} combined signal decomposition with CNN-BiGRU architectures to handle nonlinear degradation and regeneration effects, though the complexity of the model and reliance on small datasets raise concerns for real-time deployment.  

Overall, these studies highlight the potential of ML for SoH and RUL estimation, but also reveal recurring limitations: dependence on long sequences of complete cycles, lack of robustness under realistic operating conditions, and challenges in generalization across chemistries and applications. The present work advances the field by showing that accurate predictions can be achieved using only a single diagnostic cycle, thereby relaxing the data requirements and improving practical applicability in battery management systems.
