\section{Dataset}

\subsection{Dataset Description}

The dataset utilized in this study, originally presented in~\cite{sev2019}, comprises cycling data from 124 commercial lithium iron phosphate (LFP)/graphite batteries. These batteries, manufactured by A123 Systems (model APR18650M1A), have a nominal capacity of 1.1 Ah and a nominal voltage of 3.3 V. The original study aimed to optimize fast charging for lithium-ion batteries, with all batteries cycled to failure under various fast-charging conditions in a controlled environment. In this work, the dataset was accessed using the provided MATLAB struct (.mat) files, which serve as a convenient container for individual cycle data and can be loaded in MATLAB or Python (via the h5py package). Pandas dataframes can also be generated from these structs using provided code.

\subsection{Experimental Setup}

The batteries were cycled in horizontal cylindrical fixtures on a 48-channel Arbin LBT potentiostat within a forced convection temperature chamber maintained at 30°C. Charging was performed using either one-step or two-step fast-charging policies, denoted as "C1(Q1)-C2," where C1 and C2 represent the constant current rates for the first and second steps, respectively, and Q1 indicates the state-of-charge (SOC, in \%) at which the current switches. The second current step concluded at 80\% SOC, followed by a 1C constant current-constant voltage (CC-CV) charging protocol. The upper and lower voltage cutoffs were set at 3.6 V and 2.0 V, respectively, adhering to the manufacturer's specifications. These voltage limits were enforced across all charging steps, including fast charging, and after several cycles, some batteries reached the upper voltage limit during fast charging, resulting in extended constant-voltage charging periods. All batteries were discharged at a constant current rate of 4C.

\subsection{Data Measurements}

The dataset includes measurements of voltage, current, temperature, charge capacity, discharge capacity, and internal resistence for each cycle. The temperature data was collected by attaching Type T thermocouples to the exposed cell cans with thermal epoxy (OMEGATHERM 201) and Kapton tape, after removing a small section of the plastic insulation. However, the reliability of temperature measurements may vary due to inconsistent thermal contact between the thermocouple and the battery can, with some thermocouples losing contact during cycling. Internal resistance was measured during charging at 80\% SOC by averaging 10 pulses of ±3.6C, with a pulse width of 30 ms for Batches 1 and 2, or 33 ms for Batch 3.

\subsection{Batch Descriptions}

The dataset is organized into three batches, each containing approximately 48 batteries and identified sequentially as Batch 1, Batch 2, and Batch 3 based on the chronological order of their test start dates. Each batch exhibits specific experimental conditions and irregularities, detailed below.

In Batch 1, all batteries were cycled with one-step or two-step charging policies, with charging times ranging from approximately 8 to 13.3 minutes to reach 80\% SOC. Typically, two batteries were tested per policy, except for the 3.6C(80\%) policy. Rest periods of 1 minute were implemented after reaching 80\% SOC during charging, and 1-second rests followed discharging. Cycling continued until batteries reached 80\% of nominal capacity (0.88 Ah). An initial C/10 cycle was performed at the start of each test, and cutoff currents for constant-voltage steps were set to C/50 for both charging and discharging. The internal resistance test used a pulse width of 30 ms. Two automatic computer restarts during testing caused time gaps in the data, and inconsistent temperature control led to variability in the baseline chamber temperature. Tests in channels 4 and 8 failed to start, resulting in missing data, and thermocouples for channels 15 and 16 were accidentally switched. Cycle 1 data was excluded from the MATLAB struct due to an initially high sampling rate, omitted to manage file sizes. Batteries in channels 1, 2, 3, 5, and 6, using 3.6C(80\%) and 4C(80\%) policies, were paused at the end of this batch and resumed in Batch 2, leading to a capacity increase upon resumption. Tests in channels 13, 19, 21, 22, and 31 were terminated before reaching 80\% of nominal capacity.

Batch 2 involved cycling all batteries with one-step or two-step charging policies, with a fixed charging time of 10 minutes to reach 80\% SOC. Typically, one battery was tested per policy, except for the 4.8C(80\%) policy, which included three replicates. Five batteries from Batch 1 (3.6C and 4.0C policies) were resumed in this batch. Cycling continued until batteries reached 75\% of nominal capacity (0.88 Ah). Rest periods of 5 minutes were implemented after reaching 80\% SOC during charging and after discharging. An initial C/10 cycle was performed at the start of each test, with cutoff currents for constant-voltage steps set to C/50 for both charging and discharging. The internal resistance test used a pulse width of 30 ms. A computer restart around cycle 250 for most policies caused an approximately 8-hour rest period, resulting in capacity spikes. Another restart near the end of testing affected one battery (Channel 3, EL150800460623), potentially showing a capacity spike near end-of-life. Thermocouples in channels 7 and 21 detached during cycling, and those in channels 15 and 16 were switched. Data for 3.6C(80\%) and 4C(80\%) policies in channels 1, 2, 3, 5, and 6 are continuations from Batch 1, not new experiments. The battery in Channel 10 (EL150800460605) exhibited rapid degradation, possibly due to a defect, though no clear evidence was found.

In Batch 3, all batteries were cycled with two-step charging policies, with a fixed charging time of 10 minutes to reach 80\% SOC. Multiple batteries (3–8) were tested per policy. Cycling continued until batteries reached 80\% of nominal capacity (0.88 Ah). Four 5-second rest periods were implemented: after reaching 80\% SOC during charging, after the internal resistance test, before discharging, and after discharging. A final C/10 cycle was performed at 80\% of nominal capacity, with cutoff currents for constant-voltage steps set to C/20 for both charging and discharging. The internal resistance test used a pulse width of 33 ms. Some batteries experienced open-circuit voltage errors due to the internal resistance test, causing temporary pauses in cycling. Tests in channels 33 and 41 were terminated before reaching 80\% of nominal capacity. The battery in channel 46 exhibited noisy voltage profiles, likely due to an electronic connection error.

\subsection{Dataset Limitations}

The dataset, while robust for controlled experimental analysis, presents several limitations that impact the development of machine learning models for estimating the remaining useful life (RUL) and state of health (SOH) of lithium-ion batteries. All analyzed cells were cycled in a laboratory under highly controlled conditions, with continuous full charge-discharge cycles (from 0\% to 100\% state of charge and back to 0\%) and minimal idle periods between cycles. This setup facilitates systematic analysis and reproducibility but does not accurately reflect real-world operational profiles, such as those in electric vehicles or stationary energy storage systems, which often involve extended rest periods and partial, intermittent charge-discharge cycles. These differences affect both the collected signals (e.g., voltage and current profiles over time) and the cell aging process itself. As noted in the literature, cells subjected to such laboratory cycling protocols tend to exhibit shorter lifespans compared to those in real-world conditions~\cite{geslin2025}, limiting the direct applicability of developed models to operational systems.

Additionally, the dataset is constrained by its focus on a single battery chemistry (LFP/graphite) from a single manufacturer, which restricts the generalizability of machine learning models to other battery types, such as nickel manganese cobalt (NMC) or lithium cobalt oxide (LCO) chemistries, that are prevalent in various applications. The experiments were conducted at a fixed temperature of 30°C, which does not capture the impact of temperature variations commonly encountered in real-world scenarios, such as extreme heat or cold, which significantly influence battery degradation rates. The emphasis on fast-charging protocols, while valuable for studying high-rate performance, may not adequately represent slower charging scenarios or dynamic load profiles typical in practical applications, potentially limiting the models' ability to predict RUL and SOH under diverse operating conditions. Furthermore, the inconsistencies in temperature measurements due to unreliable thermocouple contacts and the presence of experimental irregularities (e.g., computer restarts, missing data, and noisy voltage profiles) introduce noise and gaps in the data, which may challenge the robustness of machine learning models, particularly those relying on precise temporal or thermal features for accurate predictions.

This dataset provides a robust foundation for applying machine learning techniques to estimate \gls{soh} and \gls{rul}, capturing a range of fast-charging conditions and their impacts on battery degradation, but its limitations necessitate careful consideration when extending models to real-world applications.

% \insertfigure{fig_model}
% \inserttable{tbl_dataset}
