\section{Dataset Descriptio}

The dataset utilized in this study, originally presented in~\cite{sev2019}, comprises cycling data from 124 commercial lithium iron phosphate (LFP)/graphite batteries. These batteries, manufactured by A123 Systems (model APR18650M1A), have a nominal capacity of 1.1 Ah and a nominal voltage of 3.3 V. The original study aimed to optimize fast charging for lithium-ion batteries, with all batteries cycled to failure under various fast-charging conditions in a controlled environment. In this work, the dataset was accessed using the provided MATLAB struct (.mat) files, which serve as a convenient container for individual cycle data and can be loaded in MATLAB or Python (via the h5py package). Pandas dataframes can also be generated from these structs using provided code.

The batteries were cycled in horizontal cylindrical fixtures on a 48-channel Arbin LBT potentiostat within a forced convection temperature chamber maintained at 30°C. Charging was performed using either one-step or two-step fast-charging policies, denoted as "C1(Q1)-C2," where C1 and C2 represent the constant current rates for the first and second steps, respectively, and Q1 indicates the state-of-charge (SOC, in \%) at which the current switches. The second current step concluded at 80\% SOC, followed by a 1C constant current-constant voltage (CC-CV) charging protocol. The upper and lower voltage cutoffs were set at 3.6 V and 2.0 V, respectively, adhering to the manufacturer's specifications. These voltage limits were enforced across all charging steps, including fast charging, and after several cycles, some batteries reached the upper voltage limit during fast charging, resulting in extended constant-voltage charging periods. All batteries were discharged at a constant current rate of 4C.

The dataset includes measurements of voltage, current, temperature, charge capacity, discharge capacity, and internal resistance for each cycle. The temperature data was collected by attaching type T thermocouples to the exposed cell cans with thermal epoxy (OMEGATHERM 201) and Kapton tape, after removing a small section of the plastic insulation. However, the reliability of temperature measurements may vary due to inconsistent thermal contact between the thermocouple and the battery can, with some thermocouples losing contact during cycling. Internal resistance was measured during charging at 80\% SOC by averaging 10 pulses of ±3.6C, with a pulse width of 30 ms for Batches 1 and 2, or 33 ms for Batch 3.

\inserttable{batch_summary}

% \insertfigure{fig_model}
% \inserttable{tbl_dataset}
