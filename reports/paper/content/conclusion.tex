\section{Conclusion}

This work investigated machine learning techniques for estimating the \acf{soh} and \acf{rul} of \acf{libs} using the large-scale dataset provided by Severson et al.~\cite{sev2019}. The dataset, comprising 124 commercial LiFePO\textsubscript{4}/graphite cells cycled under various fast-charging conditions, enabled a systematic exploration of degradation patterns and their relationship with measurable signals such as voltage, current, and temperature. Through comprehensive preprocessing and the extraction of statistical features, we established a compact yet informative representation of cell behavior that balances interpretability and computational efficiency.  

The exploratory analysis confirmed well-known degradation phenomena, including the characteristic knee point in \ac{soh} evolution, accelerated capacity fade under high charging rates, and the influence of internal resistance growth on voltage and temperature profiles. These findings provide a robust foundation for supervised learning models by linking observable features to battery aging mechanisms.  

Nevertheless, the dataset’s constraints must be acknowledged. The exclusive focus on one chemistry (LFP/graphite), highly controlled laboratory conditions, and continuous full charge--discharge cycles limit the direct applicability of the models to real-world scenarios such as electric vehicles or grid storage, where cycling is partial, irregular, and influenced by environmental variations. The inconsistencies in temperature measurements and experimental irregularities further highlight the need for robustness in model design.

Despite these limitations, the results demonstrate that statistical feature representations of single cycles can capture essential information for battery prognosis. This finding points toward a practical diagnostic framework in which SoH and RUL can be estimated from periodic diagnostic cycles, without requiring continuous monitoring of long cycle histories. Future work will extend this approach to heterogeneous datasets, incorporate physics-informed features to enhance generalization across chemistries and conditions, and evaluate the deployment of the proposed methodology in realistic battery management system environments.  

A limitation of the present approach is the requirement of a dedicated full diagnostic cycle, which may not be available in continuous operation. Future work will evaluate the feasibility of periodic diagnostics (e.g., monthly) and the extension to multiple chemistries such as NMC and NCA.
